\documentclass[12pt,a4paper,leqno]{article}

\usepackage[utf8]{inputenc}
\usepackage[T1]{fontenc}
\usepackage[english]{babel}
\usepackage{amsthm}
\usepackage{amsfonts}         
\usepackage{amsmath}
\usepackage{amssymb}
\usepackage{hyperref}
\usepackage[all]{hypcap}
\usepackage{tikz}
\usetikzlibrary{matrix, arrows, decorations.pathmorphing}

\newcommand{\R}{\mathbb{R}}
\newcommand{\C}{\mathbb{C}}
\newcommand{\Q}{\mathbb{Q}}
\newcommand{\N}{\mathbb{N}}
\newcommand{\Aff}{\mathbb{A}}
\newcommand{\Proj}{\mathbb{P}}
\newcommand{\No}{\mathbb{N}_0}
\newcommand{\Z}{\mathbb{Z}}
\newcommand{\OO}{\mathcal{O}}
\newcommand{\diam}{\operatorname{diam}}

\newcommand{\cat}{\mathcal}
\newcommand{\isomto}{\stackrel{\sim}{\rightarrow}}
\newcommand{\coker}{\mathrm{coker}}
\newcommand{\im}{\mathrm{im}}
\newcommand{\coim}{\mathrm{coim}}
\newcommand{\spec}{\mathrm{Spec}}
\newcommand{\gr}{\mathrm{gr}}
\newcommand{\plim}{\varprojlim}
\newcommand{\dlim}{\varinjlim}
\newcommand{\bl}{\mathrm{Bl}}


\newcommand{\sectionbreak}{\clearpage}

\newcommand{\fref}[1]{\hyperref[{#1}]{\ref*{#1}}}


\newtheorem{theo}{Theorem}[section]
\theoremstyle{plain}
\newtheorem{thm}[theo]{Theorem}
\newtheorem{lem}[theo]{Lemma}
\newtheorem{prop}[theo]{Proposition}
\newtheorem{cor}[theo]{Corollary}

\theoremstyle{definition}
\newtheorem{defn}[theo]{Definition}
\newtheorem{con}[theo]{Conjecture}
\newtheorem{ex}[theo]{Example}

\theoremstyle{remark}
\newtheorem{rem}[theo]{Remark}

\pagestyle{plain}
\setcounter{page}{1}
\addtolength{\hoffset}{-1.15cm}
\addtolength{\textwidth}{2.3cm}
\addtolength{\voffset}{0.45cm}
\addtolength{\textheight}{-0.9cm}

\title{Resolution of Singularities}
\author{Toni Annala}
\date{}

\begin{document}

\maketitle

\tableofcontents

\section{Introduction}

For many of the most important theorems in mathematics giving even an approximate idea of what the theorem states is fairly hard. This is not the case with resolution of singularities. The basic idea is very geometric, and can be easily stated approximately assuming very little prior mathematics background. It is a statement considering varieties, i.e., solution sets of equation systems consisting entirely of polynomial equations.

Let us start with an example. The solution set for the equation $y^2 - x^3$ is visualized in Figure \fref{cusp1}. At every point of the curve, except at the origin, the curve seems to be smooth, i.e., if we zoom close enough, it starts looking like a straight line. These kind of points are called \emph{nonsingular} or \emph{regular points}. More generally in higher dimensional varieties (say the dimension is $n$) we would consider a point to be nonsingular if locally around it the variety looks like the ''normal'' $n$-dimensional space (you can think of $\R^n$ if we are dealing with real solutions, $\C^n$ if with complex solutions).  

But something weird happens at the origin: the direction of the curve ''changes'' suddenly, and no matter how close we try to zoom, the curve will never start looking like a straight line. Points that fail to be nonsingular are called \emph{singularities} or \emph{singular points}. The type of curve singularity we have in Figure \fref{cusp1} is called a \emph{cusp}.

\begin{figure}\label{cusp1}
\begin{center}
\includegraphics{pics/cusp.png}
\caption{A cusp.}
\end{center}
\end{figure}

Now a question arises: would it be possible to parametrize our singular curve $C$ with another, nicer, curve: could we parametrize the points of $C$ with points of a nonsingular curve $C'$ (meaning that it has no singular points)? In the above case it is indeed possible. Take the curve $C'$ that is the zero set of the polynomial equation $y'^2 - x'$ plotted in Figure \fref{nsing1}. We would like to define a map $C' \to C$. Send a point $(x',y') \in C'$ to $(x',y'x')$. This point lies in $C$ as
\begin{align*}
\left( y'x' \right) ^ 2 - x'^3 = x'^2 (y'^2 - x'),
\end{align*}
and the right hand side is zero by assumption. Hence we have what we desired: a parametrization of the curve $C$ by a nonsingular curve $C'$.

\begin{figure}\label{nsing1}
\begin{center}
\includegraphics{pics/parabola.png}
\caption{The solution set for $y'^2 - x' = 0$.}
\end{center}
\end{figure}

Before going further, let us have another example. The equation $y^2 - x^3 - x^2$ gives the curve visualized in Figure \fref{node1}. The origin is clearly singular, but this singularity is different from the one we had in Figure \fref{cusp1}. The curve intersects itself! Luckily this self-intersection is in some sense as good as possible, at least the different branches meet \emph{transversally}, i.e., they have distinct tangent lines. A singular point, where two branches of a curve meet transversally is called a \emph{node}. 

Can we find a nonsingular parametrization for this curve? Yes we can. Take the curve $y'^2 - x' - 1$, which is visualized in Figure \fref{nsing2}. A parametrization is again given by $(x', y') \mapsto (x', x'y')$ which can be checked with similar reasoning as before.

\begin{figure}\label{node1}
\begin{center}
\includegraphics{pics/node.png}
\caption{This kind of singularity is called a \emph{node}.}
\end{center}
\end{figure}

\begin{figure}\label{nsing2}
\begin{center}
\includegraphics{pics/parabola2.png}
\caption{I swear, a parabola is not going to \emph{always} work.}
\end{center}
\end{figure}

Finding a parametrization of a singular variety by a nonsingular one is called \emph{resolving} its singularities (this is not what one usually means by it, usually we would require the parametrizing curve to be isomorphic to the original one almost everywhere). The problem of resolution of singularities simply asks whether or not the singularities of any variety can be resolved. It is also of some interest if there is a systematic way of doing this.

\subsection{A condenced historical account of the problem}

Although the problem is much older, we start with Oscar Zariski. Zariski was arguably the leading algebraic geometer of the first half of the twentieth century: if Grothendieck is to be regarded as the father of modern algebraic geometry, then Zariski is the grandfather. He did his undergraduate studies in Kiev, and later moved to do his doctorate in Rome under Guido Castelnuovo. During his graduate studies he was very much influenced by what is now known as the Italian school of classical algebraic geometry, which is nowdays notorious for their usage of ''geometric intuition'' leading to some false proofs and untrue theorems.

But Zariski was more algebraically inclined than his Italian teachers. And even though he enjoyed the geometric way of thinking of the Italians, he also had some worries concerning their rigour. Because of these, Castelnuovo once stated: ''Zariski, you are here with us but are not one of us'' \cite{Par}. This was not meant in a bad way, the methods of the Italian school had reached a dead-end, and would no longer be able to accomplish significant progress in the field, and Castelnuovo knew this all too well. But it would still take years before Zariski would start systematically rebuilding the foundations of algebraic geometry.

In 1932, now in the United States, Zariski began to work on his monograph \emph{Algebraic Surfaces}, a task that would take him three years. The purpose of the monograph was to give a complete account of the results concerning algebraic curves obtained by the Italian school with rigorous proofs. However, while writing the book, he became increasingly certain that the entire foundations of algebraic geometry should be completely rebuilt using commutative algebra. He later stated himself: ''working on that book took all my time because as I worked I became more and more disgusted with the kind of proofs that the Italian geometers were giving, and I started studying algebra seriously.'' \cite{Par} He started incorporating recent results from commutative algebra achieved mainly by Noether, Krull and Dedekind, and casting everything in algebraic geometry in terms of the theory of commutative rings and their ideals. 

After the completion of the monograph in 1935, Zariski went on to develop an abstract theory of algebraic geometry over an arbitrary (algebraically closed) ground field (instead of the coordinates of the points being complex numbers, they could be elements of some abstract field). It was during this time when he got interested in the resolution problem. The problem of resolution of singularities for higher dimensional varieties had long stood open, and many good mathematicians had given it a shot. Only some partial results had been achieved, and these were over complex numbers.

Zariski gave a simple proof for the resolution problem for surfaces using his notion of \emph{normalization}, a term that should ring a bell to anyone who has studied algebraic geometry or commutative algebra. Resolution for surfaces is more complicated than for curves as the singularities of a surface need not to be confined in a finite set of points: the singular points can form curves as well. Normalizing a variety produces a variety whose singularities are of codimension two, which, in the case of surfaces, means that all singularities are point-singularities, i.e., the earlier complication is now gone! After normalization, resolving the singularities was fairly straightforward.

The resolution of singularities for surfaces had, however, essentially already been solved by the Italian algebraic geometers \cite{Par}. But Zariski didn't stop there. Later, he was able to show that the singularities of three-folds (three dimensional varieties) over a field of characteristic zero (three dimensional varieties) can be resolved. This was a problem that had completely eluded the approaches of the Italian school, and was also the moment that proved to rest of the world how powerful algebraic methods could be in algebraic geometry.

This is however where the contributions of Zariski (essentially) end. For further progress some time had to pass, and another rebuilding of the foundations of algebraic geometry had to happen. Alexander Grothendieck, motivated by the famous Weil conjectures linking together number theory and geometry, began a systematic rebuilding of the whole field in mid 50s. The result was a very powerful theory of algebraic geometry, whose level of abstraction was something previously unseen in mathematics. Even Mumford, a student of Zariski who won the Fields medal in 1974, stated the following in a letter to Grothendieck: ''... I should say that I find the style of the finished works, esp. EGA, to be difficult and sometimes unreadable, because of its attempt to reach superhuman level of completeness.'' \cite{Mum} One can only imagine the struggle for those mathematicians that weren't quite as good as Mumford when trying to learn the theory. Such was the power of this theory, however, that the theory became universally accepted rather quickly.

At the time, resolution of singularities was considered one of the most important open problems in algebraic geometry, for it would enable one to transfer many questions concerning singular varieties to questions involving only nonsingular ones, which are better understood. Even Grothendieck considered the problem to be one of the two most important ones, alongside his standard conjectures (which, to this day, remain open).

In the 60s Heisuke Hironaka, another student of Zariski to be given the Fields medal, decided to attack the resolution problem. Zariski himself never truly tried to deploy the Grothendieck style of algebraic geometry. This was probably because of the great time investment required in order to learn to use it, combined with the fact that his own research was going well enough. \cite{Par} But Hironaka, like many of the students of Zariski, fully endorsed this new theory. With the new tools given by the theory of schemes, he was able to do what Zariski himself had tried and failed: in the famous paper published in 1964 in Annals of Mathematics, Hironaka gave a complete proof for resolution of singularities over a ground field of characteristic zero. 

The proof, originally around 200 pages long, quickly gained a reputation for being extremely complicated and hard to understand. However, in further research the proof has been simplified and shortened considerably. For example in \cite{Wlo} W\l{}odarczyk gives a self contained proof for resolution of singularities via a Hironaka-style argument in around twenty pages! An overview of a Hironaka-style argument for proof of resolution of singularities in characteristic 0 is given in Section \fref{HirRes}.

This still left open the finite characteristic case important in many applications to number theory. Partial results were obtained by Shreeram Abhyankar, another student of Zariski, who in 1966 published a proof for resolution of singularities for three-folds over algebraically closed fields of finite characteristic different from $2,3$ and $5$ \cite{HLOQ}. He later on went to generalize his result for three-folds in characteristic $2,3$ and $5$, as well as giving a proof for resolution of singularities for arithmetic surfaces.

However, for a long time essentially no progress was made. Hironaka's argument resisted all attempts of generalization to finite characteristic, and all algebraic approaches in the spirit of Zariski and his students seemed to fail as well. The initial success of the algebraic approach had blinded almost the entire field into thinking that this was the right way. The spell was finally broken by a young Dutch mathematician Aise Johan de Jong, who in a 1995 talk gave an outline for a very geometric proof for resolving singularities by means of so called alterations. 

His proof, published in a 1996 article \cite{Jong}, worked just as well over finite characteristic. A circle had closed: first algebra had had to replace geometry when geometrical methods failed, and now geometry, when all algebraic approaches seemed to fail, replaced algebra. Perhaps the most striking trait of his proof was the fact that essentially everything needed was already well known for nearly twenty years before the proof \cite{HLOQ}, the ingenious part was how the results were put together.

The resolution given by alterations is not, however, a resolution in the traditional sense. Traditionally one wants the resolution to produce a variety that is \emph{birational} to the original variety, i.e., it is isomorphic to the original variety outside some small subset. What de Jongs method gives is more like some kind of finite cover. But for many of the applications this is enough. The traditional problem of resolution of singularities in finite characteristic remains open.


\section{Completions}

This section is devoted to an important construction called \emph{completion} which, together with localization, forms a very powerful way of looking at the structure of an algebraic variety near a point. Familiarity with basic algebraic geometry and commutative algebra is assumed.

We begin with a motivating example. Recall the nodal curve $C$ from the first section, defined as the solution set of the polynomial $y^2 - x^3 - x^2$. At the origin the curve intersects itself, and it seems like the curve should, at least near the origin, consist of two components. But as $y^2 - x^3 - x^2$ is an irreducible polynomial, the coordinate ring $\OO_C (C)$ is an integral domain, as is the localization $\OO_{C,p}$ at origin $p$.

\begin{figure}\label{node2}
\begin{center}
\includegraphics{pics/node.png}
\caption{The nodal curve.}
\end{center}
\end{figure}

This can be thought in the following way. The open sets of the Zariski topology are just the curve $C$ minus finitely many points, so they contain almost all of the curve $C$. The localization at origin gives essentially all information about the curve $C$ found in arbitrarily small neighbourhoods of the origin, and as none of these is small enough to make $C$ irreducible, the localization stays irreducible. It is as if algebra doesn't allow us to see close enough.

Of course analytically we may break the curve into to branches: the equation of the curve
\begin{equation*}
y^2 = x^3 + x^2
\end{equation*}
can be solved to give
\begin{equation*}
y = \pm \sqrt{x^3 + x^2} =  \pm x \sqrt{1+x},
\end{equation*}
describing the two branches of the curve.

We may connect this back to algebraic geometry when we note that $\sqrt{1+x}$ can be expressed by its Taylor series. Hence, if we think $y^2 - x^3 - x^2$ as an element of the formal power series ring $k[[x,y]]$ instead of the polynomial ring $k[x,y]$, the equation $y^2 - x^3 - x^2$ cuts out a reducible subscheme of $\spec k[[x,y]]$, and the irreducible components correspond to the two branches at the origin. Somehow, when we passed into the ring of formal power series, we are now much closer to the origin than by localization. 

As $k[[x,y]]$ is a local ring with maximal ideal $(x,y)$, the localization $k[x,y]_{(x,y)}$ embeds into it, and from this it is fairly easy to see that $\OO_{C,p}$ embeds into $k[[x,y]]/(y^2 - x^3 - x^2)$. The idea of completion of local rings is exactly this: we want to get closer, so instead of just looking at the localization, we somehow replace the local ring with some sort of power series ring. We say that we pass into \emph{formal neighbourhood} of a point $p$ when we do this. The nice thing is that these are usually much simpler than the local rings: for example, if we have a nonsingular $k$-variety $X$ of dimension $d$, where $k$ is algebraically closed, then the completion $\widehat \OO_{X,p}$ of the local ring at a closed point $p$ is isomorphic to $k[[x_1,...,x_d]]$.

\subsection{Completion of topological abelian groups}

Let $G$ be a topological abelian group. We say that a sequence $(x_i)$ in $G$ is \emph{Cauchy sequence} if for all neighbourhoods $U$ of $0$ we have $N_U \in \N$ such that $x_i - x_j \in U$ for all $i,j \geq N_U$. We say that $G$ is \emph{complete} if all Cauchy sequences converge to some value in $G$. Although this notion does make sense for non-Hausdorff groups too, from now on we will require complete groups to be Hausdorff.

Before continuing, we quickly recall the following useful fact of topological groups. If we set $U_0$ to be the intersection of all neighbourhoods of 0, then it is in fact true that $U_0$ is a subgroup of $G$. Moreover, this subgroup is closed, and it is zero if and only if $G$ is Hausdorff. The topological group $G / U_0$ is Hausdorff and $U_0$ is contained in the kernel of any continuous homomorphism $G \to H$, where $H$ is Hausdorff.

\begin{defn}
Let $G$ be a topological abelian group. The \emph{(Hausdorff) completion} of $G$, denoted by $\widehat G$, is ''the smallest complete group where $G$ can be mapped into''. More precisely, this means that completion is a continuous group morphism $G \to \widehat G$, where $\widehat G$ is complete, and for all continuous homomorphisms $G \to X$, where $X$ is complete, we have a unique morphism such that the following diagram commutes:

\begin{center}
\begin{tikzpicture}[scale=1]
\node (G) at (0,0) {$G$};
\node (Ghat) at (0,2) {$\widehat G$};
\node (X) at (3,2) {$X$};


\path[]
(G) edge[->] (Ghat)
(G) edge[->] (X)
(Ghat) edge[->, dashed] node[above]{$\exists !$} (X)
;
\end{tikzpicture}
\end{center}
It is clear that this defines $\widehat G$ up to unique isomorphism if it exists. Given a continuous homomorphism $f: G \to H$, the universal property induces a morphism $\widehat f : \widehat G \to \widehat H$, and it is easy to check that this defines a functor.
\end{defn}

First property of the completion we are going to prove is the fact that it commutes with ''passing into Hausdorffication''.

\begin{prop}
Let $G$ be a topological abelian group and $U_0$ the intersection of all neighbourhoods of $0$. If the completion of $G/U_0$ exist, then $G \to G/U_0 \to \widehat{(G/U_0)}$ is the completion of $G$.
\end{prop}
\begin{proof}
Assume we have a morphism $G \to X$, where $X$ is a complete topological group. As $X$ is Hausdorff, we know that $U_0$ is contained in the kernel of our map, and we can find a unique morphism $G / U_0 \to X$ making the diagram
\begin{center}
\begin{tikzpicture}[scale=1]
\node (G) at (0,0) {$G$};
\node (GH) at (0,2) {$G/U_0$};
\node (X) at (3,0) {$X$};

\path[]
(G) edge[->] (GH)
(G) edge[->] (X)
(GH) edge[->, dashed] node[above]{$\exists !$} (X)
;
\end{tikzpicture}
\end{center}
commute. We can now use the fact that the completion of $G / U_0$ is known to exist, and hence we have a unique map $\widehat {(G/U_0)} \to X$ making the upper triangle of
\begin{center}
\begin{tikzpicture}[scale=1]
\node (G) at (0,0) {$G$};
\node (GH) at (0,2) {$G/U_0$};
\node (X) at (3,0) {$X$};
\node (GHh) at (3,2) {$\widehat{(G / U_0)}$};

\path[]
(G) edge[->] (GH)
(G) edge[->] (X)
(GH) edge[->] (GHh)
(GHh) edge[->,dashed] node[right]{$\exists !$} (X)
(GH) edge[->] (X)
;
\end{tikzpicture}
\end{center}
commute. As $G \to G/U_0$ is surjective, the map $\widehat {(G/U_0)} \to X$ is also the only one making
\begin{center}
\begin{tikzpicture}[scale=1]
\node (G) at (0,0) {$G$};
\node (Ghat) at (0,2) {$\widehat {(G/U_0)}$};
\node (X) at (3,2) {$X$};


\path[]
(G) edge[->] (Ghat)
(G) edge[->] (X)
(Ghat) edge[->, dashed] node[above]{$\exists !$} (X)
;
\end{tikzpicture}
\end{center}
commute, and hence we are done.
\end{proof}

Next we will show that the completion always exists when $G$ is first countable. The idea is exactly the same one that is used when completing metric spaces.

Denote by $s(G)$ the set of Cauchy sequences in group $G$. This is a group: if $(x_i)$ is a Cauchy sequence, then so is $(-x_i)$ for negation is homeomorphism. If $(x_i)$ and $(y_i)$ are Cauchy sequences, then so is $(x_i + y_i)$. This is not too hard to see: for all neighbourhoods $U$ of the origin, we have smaller neighbourhoods $V_1$ and $V_2$ such that $V_1 + V_2 \subset U$ by continuity of summation. Then we may choose $N$ to be large enough so that $x_i-x_j \in V_1$ and $y_i-y_j \in V_2$ for all $i,j \geq N$. Now $(x_i + y_i) - (x_j + y_j) = (x_i - x_j) + (y_i - y_j) \in U$. Similarly we can see that $s_0(G)$, the sequences of $G$ that converge to 0, is a subgroup of $s(G)$. We define a topological group $G'$ algebraically as $s(G) / s_0(G)$.

Next we define $G'$ topologically. For every neighbourhood $U$ of 0, let $U' \subset G'$ consist of those equivalence classes $[(x_i)]$ for which there exists a neighbourhood $\epsilon$ of origin, such that $x_i + \epsilon \subset U$ for $i \gg 0$. This is clearly independent of the choice of the representative $(x_i)$. This forms a neighbourhood basis for $0$ in $G'$ inducing a structure of a topological group, as the next lemma shows. 

\begin{lem}\label{TopGroupFromNbhd}
The open sets $U'$ define a structure of a topological group on $G'$ when we give $G'$ the topology where the open sets are of form
\begin{equation*}
\{O \subset G \mid \mathrm{for \ all \ } x \in G \mathrm{\ we \ have \ } U' \mathrm{\ s.t. \ } x+U' \subset O \}.
\end{equation*}
In this topology $U'$ form a neighbourhood basis for $0$ (we do not yet claim these to be open, although it is true).
\end{lem}
\begin{proof}
From basic properties of topological groups, it suffices to show that the following three properties hold:
\begin{enumerate}
\item For all $U',V'$ we have $W'$ such that $W' \subset U' \cap V'$.
\item For all $U'$ we have $V'$ such that $V'+V' \subset U'$.
\item For all $U'$ we have $V'$ such that $V' \subset -U'$.
\end{enumerate}
Verifying these conditions is fairly straightforward:

\begin{enumerate}
\item Here we may take $W = U \cap V$. For if $x \in W'$, then for some $\epsilon$ we have that $x_i + \epsilon \subset W$ for $i$ large enough, and hence also $x_i + \epsilon \subset U,V$ for such $i$.

\item Because $G$ is a topological group, we may choose a neighbourhood $V$ of the origin such that $V+V \subset U$. Assume $x,y \in V'$, so that we have $\epsilon$ such that the open sets $x_i + \epsilon$ and $y_i + \epsilon$ are contained in $V$ for $i \gg 0$. But now $(x_i + \epsilon) + (y_i + \epsilon)$ is contained in $U$, and therefore $x' + y' \in U'$. This shows that $V'+V' \subset U'$.

\item Clearly we may choose $V = -U$. 
\end{enumerate}
This finishes the proof.
\end{proof}



Next we show the continuity of our homomorphism $G \to G'$.

\begin{lem}\label{SeqCont}
The map $G \to G'$ defined earlier is continuous.
\end{lem}
\begin{proof}
It is enough to show continuity at the origin. This is shown if we show that the preimage of $U'$ is $U$. It is clear that if $x \in G$ is not in $U$, then its image in $G'$ does not lie inside $U'$. On the other hand, if $x \in U$, then by continuity of the addition, we see that we can find an open set $\epsilon$ containing $0$ such that $x + \epsilon \subset U$. Hence the image of $x$ lies in $U'$.
\end{proof}

We are getting ready to prove that $G \to G'$ is the completion. First, however, we note that the sets $U'$ are actually open, and after that we are finally able to conclude that $G'$ is complete.

\begin{lem}
The sets $U' \subset G'$ are open.
\end{lem}
\begin{proof}
Let $x \in U'$, i.e., we have an open set $\epsilon$ containing $0$ such that $x_i + \epsilon \subset U$ for large $i$. Now I claim that $x + \epsilon' \subset U'$, which would finish the proof. If $y \in x + \epsilon'$, then we have an open neighbourhood $\delta$ of origin such that $y_i - x_i + \delta \subset \epsilon$ for large $i$. But as $y_i = (y_i - x_i) + x_i$, we see that for $i \gg 0$
\begin{align*}
y_i + \delta &= x_i + (y_i - x_i) + \delta \\
&\subset x_i + \epsilon \\
&\subset U,
\end{align*}
proving the claim.
\end{proof}

\begin{lem}
If $G$ is first countable, then so is $G'$.
\end{lem}
\begin{proof}
If $(U_i)$ is a countable open neighbourhood basis for $0 \in G$, then $(U_i')$ is a countable open neighbourhood basis for $0 \in G'$, proving the claim.
\end{proof}

\begin{lem}
Assume that $G$ is first countable. Now every Cauchy sequence in $G'$ converges.
\end{lem}
\begin{proof}
Let $(U_i)$ and $(U_i')$ be as in the previous lemma. We may assume that $U_i$, and hence $U_i'$, are descending, i.e., $U_i \supset U_{i+1}$.

Let $(x^i)_i$ be a Cauchy sequence in $G'$. We can pass to a subsequence such that $x^{i_1} - x^{i_2} \in U'_n$ for $i_1, i_2 \geq n$, and to prove the convergence of the original sequence, it is enough to show that the new sequence converges. Each $x^i$ is represented by some Cauchy sequence $(x^i_j)_j$ in $G$, we may pick representative such that $x^i_{j_1} - x^i_{j_2} \in U_n$ for all $j_1, j_2 \geq n$. 

I claim that $(x^i)_i$ converges to the element $x$ of $G'$ represented by $x=(x^j_j)_j$. In order to show that this is indeed the case, we need to show that, for large $i$, we have $x^i - x \in U'_n$. Let $i$ be such that $U_i + U_i + U_i + U_i \subset U_n$. Now for all $m \geq j \geq i$ we have
\begin{align*}
x^i_j - x^j_j &= (x^i_j - x^i_m) + (x^i_m - x^j_m) + (x^j_m - x^j_j) \\
&\in U_j + U_i + U_j \\
&\subset U_i + U_i + U_i,
\end{align*}
and hence, for large $j$, we have $x^i_j - x^j_j + U_i \subset U_n$, proving that $x^i - x \in U'_n$. This concludes the proof.
\end{proof}


\begin{lem}
The group $G'$ is Hausdorff and if $G$ is Hausdorff, then $G \to G'$ is a topological embedding.
\end{lem}
\begin{proof}
Let $\mathcal U$ contain all the neighbourhoods of $0$ in $G$. To prove that $G'$ is Hausdorff, it is enough to show that $\bigcap_{U \in \mathcal U} U' = \{ 0 \}$. If $x$ lies in $U'$, for all its representatives $(x_i)$ we have that $x_i \in U$ for large $i$. Hence if $x \in \bigcap_{U \in \mathcal U} U'$, then this means that $x_i$ converges to 0, i.e., $x = 0 \in G'$.

If $G$ is Hausdorff and $x \not = y$ are two of its elements, then $x-y$ is nonzero, and we have a neighbourhood $U$ of the origin such that $x-y \not \in U$. Therefore the images of $x$ and $y$ in $G'$ cannot coincide, and we can conclude that $G \to G'$ is an injection. 

Finally, we need to show that the map $G \to \im (G)$ is a homeomorphism. But we have done the essential work already: by the proof of \fref{SeqCont} it is clear that the image of an open neighbourhood $U$ of the origin is just $U' \cap \im (G)$.
\end{proof}

Now we are finally ready to prove the existence of completion.

\begin{thm}
If $G$ is first countable then $G \to G'$ is the completion of $G$.
\end{thm}
\begin{proof}
At least we know that $G'$ is complete. Assume then that $X$ is a complete topological group, and $f: G \to X$ a continuous homomorphism. It is clear by the definition of $G'$ that it's elements can be approximated by a sequences contained in the image of $G$. If we want to extend $f$ into a continuous function $f' : G' \to X$, then for a point $x \in G'$ represented by a Cauchy sequence $(x_i)$ in $G$, we must have that $f' (x) = \lim f(x_i)$. We are done if we can show that $f'$ defined as above gives a well defined continuous morphism $G' \to X$.

First of all, as $f: G \to X$ is continuous, it is easy to see that it sends Cauchy sequences in $G$ to Cauchy sequences in $X$. Hence at least the limit $\lim f(x_i)$ exists. If the Cauchy sequences $(x_i)$ and $(x'_i)$ define the same element of $G$, then $x_i - x'_i \to 0$, and thus $f(x_i) - f(x'_i) = f(x_i - x'_i) \to 0$. This proves that our map $f'$ is a well defined function $G' \to X$. It is also easy to see that this is a homomorphism of groups. The final thing left is to show that $f'$ is continuous.

Let $V \subset X$ be an open neighbourhood of $0$. Let $0 \in V_2 \subset V$ be an open set such that $V_2 + V_2 \subset V$, and $U_2$ the preimage of $V_2$ in $G$. I claim that $f'$ sends $U_2'$ into $V$. This is almost trivial: if $x$ represented by $(x_i)$ lies in $U_2'$, then for large $i$, $x_i \in U_2$, and hence $f(x_i) \in V_2$. But from this it follows that $f(x_i) + V_2 \subset V$ for $i \gg 0$ and we can conclude, using the next lemma, that $f'(x) \in V$, which is exactly what we wanted. 
\end{proof}

\begin{lem}
If $(x_i)$ is a convergent sequence in a topological abelian group $G$ such that for some open set $\epsilon$ containing $0$ $x_i + \epsilon \subset V$ for $i \gg 0$, then $x_i \to x \in V$.
\end{lem}
\begin{proof}
As $x_i \to x$, we must have $x - x_i \in \epsilon$ for large $i$, from which it follows that $x \in x_i + \epsilon \subset V$.
\end{proof}

This general construction simultaneously takes care of, for example, completing rational numbers in the usual topology to obtain real numbers $R_p$, in the $p$-adic topology to obtain $p$-adic numbers $\Q_p$, completing $k[x_1,...,x_n]$ in a specific topology to obtain $k[[x_1,...,x_n]]$ and much more. We gather here some now trivial properties of the completion.

\begin{cor}
The completion of a discrete topological group $G$ is $G$.
\end{cor} 

\begin{cor}
Let $H$ be a subgroup of first countable Hausdorff $G$. Now the closure of $H$ in $\widehat G$ is the completion of $H$.
\end{cor}
\begin{proof}
Now $H$ is first countable and Hausdorff, and using the construction for the completion we just have described, the claim is trivial, as the closure of $H$ in $\widehat G$ is the set of points that can be approximated by sequences in $H$ ($\widehat G$ is first countable, otherwise this would not necessarily hold).
\end{proof}

\subsubsection*{Filtrations}

A special kind of topology often encountered in algebra is one given by \emph{filtrations}. If $G$ is a topological abelian group, then a filtration is just a descending chain 
\begin{equation*}
G = G_0 \supset G_1 \supset G_2 \supset ...
\end{equation*}
of subgroups. The filtration induces a structure of a topological group, where $G_i$ forms a neighbourhood basis for $0$ (the argumentation is almost identical to \fref{TopGroupFromNbhd}). As the $G_i$ are subgroups, for all $x \in G_i$ we have $x + G_i \subset G_i$, and therefore the sets $G_i$ will be open in the topology.

$G$ is necessarily first countable, and Hausdorff if and only if $\bigcap_i G_i$ is the trivial subgroup. Assuming this is the case, we can give another construction for the completion of $G$. It is based on the fact that for every equivalence class of Cauchy sequences, one can choose a representative $(x_i)$ such that $x_i - x_j \in G_n$ for all $i,j \geq n$. Therefore $x_i$ defines the following values up to $G_i$. Such sequences are clearly in one to one correspondence with elements of $\prod_i (G/G_i)$ where the higher coordinates agree with the $i^{th}$ one modulo $G_i$. This is actually a special case of a general construction known as the \emph{projective limit}. Usually this is denoted by $\varprojlim (G/G_i)$.

This construction is actually quite useful, as we shall see in a moment. Define a map $\Delta_G : \prod_i (G/G_i) \to \prod_i (G/G_i)$ by sending $([x_i])$ to $([x_i - x_{i+1}])$. The following proposition is just a restatement of the definition.

\begin{prop}
The limit $\varprojlim (G / G_i)$ is exactly the kernel of $\Delta_G$.
\end{prop}

Using the above proposition, we can obtain some nice results rather easily using just a tiny bit of basic homological algebra. Namely:

\begin{thm}\label{ExactCompletion}
Let $H$ be a (topological) subgroup of $G$. Now we have a canonical isomorphism
\begin{equation*}
\plim (\overline{G} / \overline{G_i}) \cong \plim (G / G_i) / \plim (H / H_i), 
\end{equation*}
where $\overline G = G/H$, $\overline G_i$ is the image of $G_i$ in $\overline G$, and $H_i = H \cap G_i$.
\end{thm}
\begin{proof}
For each $i$, we clearly have the exact sequence
\begin{equation*}
0 \to H / H_i \to G / G_i \to \overline{G} / \overline{G_i} \to 0
\end{equation*}
This gives rise to the following commutative diagram with exact rows:
\begin{center}
\begin{tikzpicture}[scale=3]
\node (01) at (0.3,0) {$0$};
\node (02) at (3.7,0) {$0$};
\node (03) at (0.3,-0.5) {$0$};
\node (04) at (3.7,-0.5) {$0$};
\node (H1) at (1,0) {$\prod_i (H / H_i)$};
\node (G1) at (2,0) {$\prod_i (G / G_i)$};
\node (Gb1) at (3,0) {$\prod_i (\overline{G} / \overline{H_i})$};
\node (H2) at (1,-0.5) {$\prod_i (H / H_i)$};
\node (G2) at (2,-0.5) {$\prod_i (G / G_i)$};
\node (Gb2) at (3,-0.5) {$\prod_i (\overline{G} / \overline{H_i})$};

\path[]
(01) edge[->] (H1)
(H1) edge[->] (G1)
(G1) edge[->] (Gb1)
(Gb1) edge[->] (02)
(03) edge[->] (H2)
(H2) edge[->] (G2)
(G2) edge[->] (Gb2)
(Gb2) edge[->] (04)
(H1) edge[->] node[right]{$\Delta_H$} (H2)
(G1) edge[->] node[right]{$\Delta_G$} (G2)
(Gb1) edge[->] node[right]{$\Delta_{\overline{G}}$} (Gb2)
;
\end{tikzpicture}
\end{center}
The snake lemma gives us the exact sequence
\begin{equation*}
0 \to \plim (H / H_i) \to \plim (G / G_i) \to \plim (\overline{G} / \overline{G_i}) \to \coker(\Delta_H)
\end{equation*}
which proves our claim, as $\coker(\Delta_H) = 0$ by the next lemma.
\end{proof}

\begin{lem}
The map $\Delta_G$ is surjective. 
\end{lem}
\begin{proof}
Let us have an element $x = (x_i) \in \prod_i (G / G_i)$. We construct an element $x' = (x'_i)$ that maps to $x$. Pick $x'_1 = x_1$  and $x'_2 = 0$, so that we have an element $(x'_1, x'_2, 0, 0, ...)$ mapping to $(x_1, x'_2, 0, ...)$.

Assume we have chosen $(x'_1, ..., x'_{n+1}, 0 , ...)$ mapping to $(x_1,...,x_n,x'_{n+1},0,...)$. As the map $G / G_{n+2} \to G / G_{n+1}$ is surjective, we may choose $x'_{n+2}$ in such a way that the difference $x'_{n+1} - x'_{n+2}$ equals $x_{n+1}$. But now $(x'_1, ..., x'_{n+2},0 , ...)$ maps to $(x_1,...,x_{n+1}, x'_{n+2},0,...)$ and we are done by induction.
\end{proof}

We have the following immediate corollary:

\begin{cor}
Let $G$ be a topological group, whose topology is given by a filtration, and $H \leq G$ a subgroup. Now we have a canonical isomorphism 
\begin{equation*}
\widehat{(G/H)} \cong \widehat{G} / \widehat{H}.
\end{equation*}
\end{cor}
\begin{proof}
This is a trivial consequence of the previous discussion. The fact that $G$ is Hausdorff is used for identifying $\plim (G/G_i)$ with the completion, and the fact that $H$ is closed is necessary to make sure that $G/H$ is Hausdorff.
\end{proof}

This looks a bit like the completion might be exact, at least in good cases. But it is quite dangerous to try to think about these properties in that way: the category of topological abelian groups is not abelian as there is too much freedom in the choice of topology, and therefore one has to be quite careful when talking about exact sequences. This can, however, be fixed by restricting into a suitable subcategory which is restrictive enough on topology.

Quite often, a convenient way of specifying an element of $\widehat G$, where $G$ has the topology induced by a filtration, is to give an infinite sum
\begin{equation*}
\sum_i x_i
\end{equation*}
where $x_i \in G_i$. This is interpreted as the limit of the Cauchy sequence formed by the partial sums. It is clear that every element of the completion $\widehat G$ can be expressed in this form. 

\subsection{Completions of topological rings and modules}

A \emph{topological ring} $A$ is a topological abelian group with a continuous multiplicaiton map $A \times A \to A$ satisfying the ring axioms. If $A$ is a topological ring, then, quite naturally, a \emph{topological $A$-module} is a topological abelian group together with a continuous multiplication $A \times M \to M$. We define the \emph{completion} of a topological ring and a topological module to be the completion as a topological abelian group.

Using the constructions we have for completion, it is easy to see that if $A$ is Hausdorff and first countable, then $\widehat A$ is a topological ring, and similarly if $M$ is Hausdorff and first countable, then $\widehat M$ is both a topological $A$ and a topological $\widehat A$-module.

If $A$ is a ring ($M$ an $A$-module), then a \emph{filtration} on $A$ (on $M$) is a descending sequence $I_1 \supset I_2 \supset ...$ of ideals (descending sequence $M_1 \supset M_2 \supset ...$ of submodules). Again, it is easy to see that filtration on $A$ gives it a structure of a topological ring, and a filtration $(M_i)$ on $M$ satisfying $I_j M_i \subset M_{i+j}$ gives rise to a structure of a topological $A$-module. 

The most common filtration occurring in commutative algebra is the \emph{$I$-adic} filtration: we have an ideal $I$ of $A$, and we set $I_i = I^i$. Similarly we have the $I$-adic filtration on $M$ defined by $M_i = I^i M$. The induced topology is called the \emph{$I$-adic topology}. Two well known theorems in commutative algebra can be given a topological meaning.

\begin{prop}
\textbf{(Topological Krull's intersection theorem)} Let $A$ be a local Noetherian ring and $I$ a proper ideal. Now the $I$-adic topology on $A$ and on finite $A$-modules $M$ is Hausdorff.
\end{prop}
\begin{proof}
The usual statement is that the intersection $\bigcap_i I^iM$ is zero, which is equivalent to the filtration inducing a Hausdorff topology.
\end{proof}

\begin{prop}
\textbf{(Topological Artin-Rees lemma)} Let $A$ be a Noetherian ring and $I$ its proper ideal. Give the ring $A$ the $I$-adic topology. If $M$ is a finitely generated $A$-module with $I$-adic topology and $N$ is a submodule of $M$, then the topology $N$ inherits from $M$ is exactly the $I$-adic topology. 
\end{prop}
\begin{proof}
Again, the usual statement is just that the $I$-adic filtration on $M$ restricts to a $I$-good filtration on $N$, which is easily seen to prove the claim.
\end{proof}

Assume that the ring $A$ is Noetherian. We can now restrict our attention to the \emph{category of finitely generated $I$-adic $A$-modules}. The objects are just finitely generated $A$-modules with the $I$-adic topology and the morphisms are the continuous $A$-linear maps. However, if $f: M \to N$ is \emph{any} map of $A$-modules, then $f(I^iM) \subset I^i N$, and hence the morphism is seen to be continuous in the $I$-adic topology. Hence, categorically, the finitely generated $I$-adic $A$-modules are equivalent to finitely generated $A$-modules. Especially we can conclude that they form an abelian category.

Now we can finally interpet the proof of \fref{ExactCompletion} as saying that the completion is an exact functor. But for future use, it is convenient to note that the completion coincides with taking the tensor product with $\widehat A$. As the completion $M \to \widehat M$ is a map to a $\widehat A$-module, we have the usual map $\widehat A \otimes_A  M \to \widehat M$. This is easily seen to define a natural transformation, which in good situations is an isomorphism.

\begin{prop}
Let $A$ be a Noetherian ring and $M$ a finitely generated $A$-module. Now the natural map $\widehat A \otimes_A M \to \widehat M$ is an isomorphism.
\end{prop}
\begin{proof}
This is clear if $M$ is a free module. In the general case, as $A$ is Noetherian, we know that $M$ is \emph{finitely presented}, i.e., we have an exact sequence
\begin{equation*}
F_1 \to F_0 \to M \to 0
\end{equation*}
where the modules $F_i$ are free. Using what we already know for free $A$-modules, we obtain the commutative diagram
\begin{center}
\begin{tikzpicture}[scale=1.5]
\node (AF1) at (0,1) {$\widehat A \otimes_A F_1$};
\node (AF0) at (2,1) {$\widehat A \otimes_A F_0$};
\node (AM) at (4,1) {$\widehat A \otimes_A M$};
\node (01) at (5.5,1) {$0$};

\node (F1h) at (0,0) {$\widehat F_1$};
\node (F0h) at (2,0) {$\widehat F_0$};
\node (Mh) at (4,0) {$\widehat M$};
\node (02) at (5.5,0) {$0$};

\path[]
(AF1) edge[->] (AF0)
(AF0) edge[->] (AM)
(AM) edge[->] (01)
(F1h) edge[->] (F0h)
(F0h) edge[->] (Mh)
(Mh) edge[->] (02)
(AF1) edge[->] (F1h)
(AF0) edge[->] (F0h)
;
\end{tikzpicture}
\end{center}
where the vertical maps are isomorphisms. Now we can use the uniqueness of cokernels to conclude that the natural map $\widehat A \otimes_A M  \to \widehat{M}$ is an isomorphism.
\end{proof}

In the process we have seen that $\widehat A$ is a flat $A$-algebra.

\begin{prop}
Assume that $A$ is a Noetherian ring. Now the completion $\widehat A$ is a flat $A$-algebra.
\end{prop}
\begin{proof}
As tensoring with $\widehat A$ is the same as completion for all \emph{finitely generated} $A$-modules, we see that $\widehat A \otimes_A $ preserves all short exact sequences of finitely generated $A$-modules. But it is well known that this is equivalent to $\widehat A$ being flat.
\end{proof}

From now on, assume that we are dealing with a local ring $A$ with maximal ideal $m_A$ and residue field $K = A / m_A$. Moreover, unless otherwise stated, all completions will be taken with respect to the $m_A$-adic topology. 

\begin{prop}
The completion $\widehat A $ is a local ring with maximal ideal $\widehat{m_A}$ and residue field $K$.
\end{prop}
\begin{proof}
We know that $\widehat A / \widehat{m_A} \cong \widehat{(A/m_A)}$ naturally, and as $A / m_A$ is $K$ with the discrete topology, the left hand side of the equation is $K$. This proves that $m_A$ is \emph{a} maximal ideal. 

In order to see that it is the only one, we need to show that $\widehat {m_A}$ contains all nonunits of $\widehat A$. This is perhaps most easily seen from the limit construction: the completion of $m_A$ consists of exactly those elements of $\plim (A / m_A^i)$, whose first coordinate is zero. If this is not the case, then in all coordinates we have something in $A / m_A^i$ not lying in the image of $m_A$. But this means that the coordinates are units, and hence the element is invertible as well, proving that all elements lying outside $\widehat{m_A}$ are units.
\end{proof} 

\subsection{Hensel's lemma}

One of the nicest property of complete rings is what is known as the Hensel's lemma. The version we are going to give states that if a polynomial $f \in A[x]$ has a simple root in $A/m_A$, then it also has a root in $\widehat A$.

\begin{thm} \textbf{Hensel's lemma.} Let $f$ be a polynomial in $A[x]$ which has a simple root $[a_0] \in A/m_A$. Now there is an element $a \in \widehat A$ such that $a = a_0$ modulo $m_A$ and $f(a) = 0$.
\end{thm} 
\begin{proof}
It is enough to find sums $a_0 + a_1 + ... + a_n$, where $a_i \in m_A^i$, such that $f(a_0 + ... + a_n) \in m_A^{n+1}$ for all $n$. This is because the polynomial $f$ clearly defines a continuous map $A \to A$ and $f(a_0 + ... + a_n) \to 0$ as $n$ tends to infinity. Thus, if we set $a = a_0 + a_1 + ...$, then $f(a) = 0$.
 
Assume we have already chosen $a_0,...,a_n$, and would like to choose $a_{n+1} \in m_A^{n+1}$ in a way that our inductive assumption would still hold. Now we have
\begin{equation*}
f(a_0 + ... + a_n + a_{n+1}) = f(a_0 + ... + a_n) + f'(a_0 + ... + a_n)a_{n+1} + b a_{n+1}^2
\end{equation*}
where $b \in A$. If we look at the right hand side in $A/m_A^{n+2}$, we obtain the following equation:
\begin{equation*}
0 = f(a_0 + ... + a_n) + f'(a_0 + ... + a_n)a_{n+1}.
\end{equation*}
now $f'(a_0 + ... + a_n) = f'(a_0) \not = 0$ in $A/m_A$, and hence $f'(a_0 + ... + a_n)$ is invertible in $A/m_A^{n+2}$. We may now solve $a_{n+1}$ from the equation:
\begin{equation*}
a_{n+1} = - {f(a_0 + ... + a_n) \over f'(a_0 + ... + a_n)},
\end{equation*}
and we are done by induction.
\end{proof}

\subsection{Structure of completetions of local Noetherian rings}

One of the great things about completion is how it simplifies the theory of local rings. Although localization tries to encapsulate only the local information near a point in a variety, it usually contains very much information, as the open sets of the Zariski topology are so large. Therefore the structure of the local rings is also quite varied. However, once we require completeness, the rings have surprisingly little structural freedom. In this subsection we concentrate on local $k$-algebras where $k$ is a field of characteristic 0.

\begin{lem}
Let $A$ contain a field of characteristic 0. Now $\widehat A$ contains a coordinate field, i.e., a field mapping isomorphically onto $ A/ m_A$ in the quotient map $\widehat A \to \widehat A/ \widehat {m_A} = A / m_A$.
\end{lem}
\begin{proof}
Clearly $A$ contains a field of characteristic 0 if and only if it contains $\Q$. Denote by $K$ the residue field $A/m_A$. 

Let $(\alpha_i)_{i \in I}$ be a transcendence basis for $K$ over $\Q$, i.e., a collection of elements of $K$ satisfying no nontrivial algebraic relations over $\Q$ such that $K$ is algebraic over $\Q((\alpha_i)_{i \in I})$. Now we can clearly choose elements $a_i \in A$ such that $a_i = \alpha_i$ modulo $m_A$, and thus we see that $\Q((\alpha_i)_{i \in I}) \cong \Q((a_i)_{i \in I}) \subset A$. Hence $\Q((a_i)_{i \in I})$ is also contained in $\widehat A$.

Now we may extend $\Q((a_i)_{i \in I})$ into $K$ by using Hensel's and (if needed) transfinite induction. This works because if we have an intermediate field extension $\Q((a_i)_{i \in I}) \subset M \subset K$, then also $M$ has characteristic $0$, and hence any element of $K$ is a simple root of a polynomial with coefficients in $M$, and hence can be lifted to $\widehat A$ by Hensel's lemma.
\end{proof}

To make use of the previous result, we introduce a useful construction. Given a (not necessarily local) ring $A$, we can construct the \emph{graded ring} associated to $I$-adic filtration, denoted by $\gr_I (A)$ or just $\gr (A)$, as follows: as an abelian group, it is just the direct sum $\bigoplus_{i=0}^\infty (I^i / I^{i+1})$. To define it as a graded ring, we simply note that the map $(I^i / I^{i+1}) \times (I^j / I^{j+1}) \to I^{i+j} / I^{i+j+1}$ induced by the multiplication of $A$ is well defined. 

Assume that we have a morphism of $f: A \to B$ of general commutative rings, and let us have ideals $I \subset A$ and $J \subset B$ such that $fI \subset J$. Now also $fI^i \subset J^i$, and therefore we may define a map $\gr_I (A) \to \gr_J (B)$ of graded rings.

\begin{lem}
Let us have a homomorphism $f: A \to B$ of rings, and ideals $I \subset A$ and $J \subset B$ be such that $fI \subset J$ and the induced map $A / I \to B / J$ is a surjection. If the image of $I$ generates $J$ over $B$, then the associated map $\gr (A) \to \gr (B)$ is surjective.
\end{lem}
\begin{proof}
It is enough to show that for all $i$, the degree $i$ part of the associated map is surjective. The case $i=0$ has already been taken care of as $A/I \to B/J$ was assumed to be surjective. By the assumption that $f I$ generates $J$ over $B$, we see that $f I^i$ generates $J^i$ over $B$. Hence the image of $I^i / I^{i+1}$ generates $J^i / J^{i+1}$ over $B/J$. But as $J^i / J^{i+1}$ can be also thought as an $A/I$ algebra, and as $A/I \to B/J$ is surjective, and finally as the associated map $I^i / I^{i+1} \to J^i / J^{i+1}$ is a map of $(A/I)$-modules, we see that the map $I^i / I^{i+1} \to J^i / J^{i+1}$ must be surjective.
\end{proof}

Using the above lemma, we can prove an important theorem concerning the structure of completions.

\begin{thm}
Let $A$ be a Noetherian local ring containing a field of characteristic 0. Now $\widehat A$ is isomorphic to $K[[x_1,...,x_n]]/I$, where $K = A/m_A$.
\end{thm}
\begin{proof}
As $A$ is Noetherian, the maximal ideal $m_A$ is generated by a finite collection $a_1,...,a_n$ of elements. Moreover the images of these elements in $\widehat A$ generate the maximal ideal $\widehat{m_A}$. As $K$ is contained in $\widehat A$, we get a map of rings $K[x_1,...,x_n] \to \widehat{A}$ which sends $x_i$ to $a_i$. If we denote by $I$ the ideal of $K[x_1,...,x_n]$ generated by all the $x_i$, then we can use the above lemma to see that the associated map $\gr_I(K[x_1,...,x_n]) \to \gr_{\widehat{m_A}}(\widehat{A})$ is surjective. But this means that $K[x_1,...,x_n] / I^i \to \widehat{A} / \widehat{m_A}^i$ is surjective for all $i$. 

We notice that $\plim (K[x_1,...,x_n] / I^i) = K[[x_1,...,x_n]]$ and $\plim (\widehat{A} / \widehat{m_A}^i) = \widehat A$, and hence we are done using the next lemma.
\end{proof}

\begin{lem}
Let $G \to H$ be a map of abelian groups with filtrations $(G_i)$ and $(H_i)$ respectively. Assume that the image of $G_i$ is contained in $H_i$, i.e., that we have induced maps $G / G_i \to H / H_i$. If these induced maps are surjective, then so is the limit map $\plim (G/G_i) \to \plim (H / H_i)$.
\end{lem}
\begin{proof}
This is essentially proven in \fref{ExactCompletion}.
\end{proof}

One of the nicer theorem concerning completions tells us, among other things, that if we have a regular point $p$ on a variety $X$, then formally locally the variety looks like an affine space near the point $p$, i.e., $\widehat \OO_{X,p} \cong K[[x_1,...,x_n]]$. Before proving this, however, we need a lemma.

\begin{lem}
Let $A$ be a regular local ring of dimension $d$, $K = A / m_A$,  and let $m_A$ be generated by a regular sequence $a_1,...,a_d$. Now the associated graded ring is isomorphic to $K[x_1,...,x_d]$ via a map $K[x_1,...,x_d] \to \gr_{m_A} (A)$ sending $x_i$ to the image of $a_i$ in $m_A / m_A^2$.
\end{lem}
\begin{proof}
As $a_i$ generate the maximal ideal of $A$, we see that the map $k[x_1,...,x_d] \to \gr (A)$ is surjective (it is clear that $\gr_I (k[x_1,...,x_d])$, where $I  = (x_1,...,x_d)$, is just $k[x_1,...,x_d]$ with the usual grading). We need to show that this is an injection, which is equivalent to stating that for all nonzero homogeneous polynomials $F$ we have $F (a_1,...,a_d) \not = 0$.

Assume that this is not the case, i.e., we have some nonzero homogeneous $F$ of degree $n$ such that $F(a_1,...,a_d) = 0$. We can lift the coefficients of $F$ from $K$ to $A$ and obtain a homogeneous polynomial $\widetilde F \in A [x]$ such that $\widetilde F (a_1,...,a_d) \in m_A^{n+1}$. Thus for some homogeneous polynomial $H$ of degree $n+1$, we have that $(\widetilde F - H) (a_1,...,a_d) = 0$, i.e., we have a nontrivial algebraic relation, defined by a polynomial $f \in A[x]$ with no constant terms, among the $a_i$. 

But this leads to a contradiction. If $x_1$ divides $f$, we are done, as $a_1$ is a zerodivisor. Otherwise we move to modulo $a_1$ and to a new polynomial $f(0,x_2,...,x_d)$. Then we do the same thing for $a_2$ and $a_3$ and so on, until we find that some $a_{i}$ is a zerodivisor modulo $a_1,...,a_{i-1}$. This must happen at least when $i = d$, as the polynomial $f$ had no constant terms. This concludes the proof.
\end{proof}

\begin{thm}
Let $A$ be a regular local ring of dimension $d$. Now there is an isomorphism $K[[x_1,...,x_d]] \isomto \widehat A$, where $K = A/m_A$, and $x_1,...,x_d$ is sent to an image of a regular sequence $a_1,...,a_d$ generating $m_A$ in $\widehat A$.
\end{thm}
\begin{proof}
Let $a_1,...,a_d$ be a regular sequence generating the maximal ideal $m_A$ of $A$. By the previous lemma we have that the map $\gr_I (K[x_1,...,x_d]) \to \gr_{m_A} (A)$, where $I = (x_1,...,x_d)$, associated to the map sending $x_i$ to $a_i$ is an isomorphism. By dimensionality, we can therefore conclude that $K[x_1,...,x_d] / I^i \to A/m_A^i$ is an isomorphism for all $i$, which shows that the respective completions are isomorphic as well, proving the theorem.
\end{proof}

\section{Blowing Up}\label{BlowUp}

One of the most common ways of getting rid of singularities is called \emph{blowing up}. The simplest case is blowing up a point on a plane, which we are going to introduce through an example now. Recall our old friend, the nodal curve, which has a singularity at the origin. We are going to get rid of the singularity by blowing up the origin.

\begin{figure}\label{node3}
\begin{center}
\includegraphics{pics/node.png}
\caption{The nodal curve.}
\end{center}
\end{figure}

We begin by mapping each point of the curve, except the origin, into a three dimensional space via the map that sends $(x,y)$ to $(x,y,y/x)$. The $z$-coordinate is just the slope of the direct line connecting the point $(x,y)$ with the origin.

\begin{figure}\label{blownUpNode}
\begin{center}
\includegraphics[scale=0.8]{pics/blown_up_node.png}
\caption{The Zariski closure of the image of nodal curve in affine 3-space.}
\end{center}
\end{figure}

When we embed the nodal curve, except of course the origin, this way into the affine three space $\Aff^3$, the resulting space curve won't have any points on the locus $x=0$. The points $(x,y,z)$ belonging to the image satisfy two algebraic relations. First of all, they must satisfy $y^2 - x^3 - x^2 = 0$ for this was the defining equation for the nodal curve. Secondly, we must have $zx = y$ because we sent the point $(x,y)$ of the curve to $(x,y,y/x)$.

Any point of $\Aff^3$ having $x=0$ and $y=0$ satisfy the two algebraic relations we just mentioned, and in fact, if we denote by $C'$ the set of points in $\Aff^3$ lying in the image of the nodal curve without origin, we can see that the two polynomial equations cut out exactly $C'$ and the $z$-axis. We can do better. The \emph{Zariski closure} $\overline{C'}$ of $C'$ in $\Aff^3$ is the smallest subset of $\Aff^3$ that is given by some polynomial equation system and that contains $C'$. In this case it is the set of points cut out by $zx - y$ and $z^2 - x - 1$, and it is visualized in \fref{blownUpNode}. Projection $\Aff ^3 \to \Aff^2$ sending $(x,y,z) \to (x,y)$ maps the Zariski closure $\overline{C'}$ to the nodal curve $C$, and this mapping $\overline{C'} \to C$ is called the \emph{blow-up}.

Outside the origin the preimage of a point of the nodal curve $C$ is a single point in $\overline{C'}$, and the preimage of the origin consist of two points that correspond to the two branches. The blown up curve $\overline{C'}$ contains no singularities, so we have managed to resolve them. In this section, we are going to give the basic constructions and results of blow-ups.

\subsection{Blowing up points in an affine space}

Blowing up a point in an affine space $\Aff^n$ is essentially very similar to what happened at the beginning of this section, although it is a bit more technical. We are only going to describe how to blow up the origin, the technique trivially generalizes to other points as well. From now on, unless otherwise stated, we work over an algebraically closed field.

The basic idea is to replace the origin with $\Proj^{n-1}$ whose points corresponds to all the ''tangent directions'' at 0. We begin by taking the product space $\Aff^n \times \Proj^{n-1}$, whose coordinates we are going to denote by $(x_1,...,x_n, [u_1:...:u_n])$. As we would like to have only the points of form $(x_1,...,x_n, [x_1:...:,x_n])$, we set the blown up space $\bl_0 (\Aff^n)$ (where 0 stands for the fact that we blow up the origin) to be the closed subvariety of $\Aff^n \times \Proj^{n-1}$ cut out by equations $x_i u_j - x_j u_i = 0$. The first projection map clearly induces a morphism $\bl_0 (\Aff^n) \to \Aff^n$ that is isomorphism outside the origin of $\Aff^n$. Over the origin we have the whole projective $(n-1)$-space, so we have managed to do exactly what we wanted.

It is however more interesting to see what happens to \emph{subvarieties} $X$ of $\Aff^n$ under the blow-up. The preimage of $X$ in the map $\bl (\Aff^n) \to \Aff^n$ is called the \emph{total transform} of $X$, and it is denoted by $X^{tot}$. The Zariski closure of the preimage of $X \backslash \{ 0 \}$ in $\bl (\Aff^n)$ is called the \emph{strict transform} $X^{st}$ of $X$, and taking the strict transform is what is usually meant when we blow up a point on $X$. It is clear that the blow-up map $\bl (\Aff^n) \to \Aff^n$ restricts to give morphisms $X^{tot} \to X$ and $X^{st} \to X$.

\begin{ex}
Lets get back to the example of the nodal curve $C$ cut out of $\Aff^2$ by the polynomial $y^2 - x^3 - x^2$. We denote the points of the product variety $\Aff^2 \times \Proj^2$ by $(x,y, [u:v])$, so the blown up plane $\bl (\Aff^2)$ is cut out by the polynomial $xv - yu$. Now the preimage of $C$ in $\bl (\Aff^2)$ is cut out by furthermore requiring the equation $y^2 - x^3 - x^2$ to be satisfied. It is easy to see that if $(x,y) \not =  (0,0)$, then $(x,y,[x:y])$ is the only point of $C^{tot}$ lying over it, and as every point of form $(0,0, [u:v])$ is in $C^{tot}$, we see that the preimage of the origin is the whole $\Proj^1$.

We can compute the strict transform of $C$ locally. We pass to the locus where $u \not = 0$, so that we can parametrize the points by $(x,y,z)$ where $z = v/u$. The part of the blown up plane lying here is cut out by $xz - y$. Now we notice that, as an algebraic variety, this is isomorphic to affine 2-space $\Aff^2$ by the map $(x,z) \mapsto (x,xz,z)$. The preimage of $C$ minus the origin in this latter affine 2-space is the zero set of $x^2 z^2 - x^3 - x^2$ outside the locus $x=0$ (if $x=0$, then also $xz=y$ is 0). The smallest algebraic set containing the preimage is the one cut out by $z^2 - x - 1$, because the resulting algebraic set must be one dimensional, and because the curve defined by the above equation is irreducible. 

We conclude that in the first chart of $\Aff^2 \times \Proj^1$ the strict transform of $C^{st}$ is cut out by the equations $xz - y$ and $z^2 - x - 1$, and hence the equations $xv - yu$ and $v^2 - (x+1)u^2$ almost cut out the strict transform in $\Aff^2 \times \Proj^1$. As there may still be some extra points in the locus $u=0$, we compute the strict transform in the second chart of $\Aff^2 \times \Proj^1$ and see that the equation $1 - y z^{-3} - z^{-2}$. Therefore we obtain one additional equation $v^3 - yu^3 - u^2 v = 0$. Hence the strict transform $C^{st} \subset \Aff^2 \times \Proj^1$ is given by the equations
\begin{align*}
xv - yu &= 0 \\
v^2 - (x+1)u^2 &= 0 \\
v^3 - yu^3 - u^2 v &= 0.
\end{align*}
\end{ex}

\begin{ex}
The procedure given in the last example easily generalizes to computing the strict transformation of a hypersurface $X \subset \Aff^n$ when blowing up the origin. Let $f$ be the defining polynomial for $X$, and let $d$ be the degree of its lowest term. For the $i^{th}$ chart of the $\Aff^{n} \times \Proj^{n-1}$ we get the relations 
\begin{equation*}
x_j = {u_j \over u_i} x_i  
\end{equation*}
and the polynomial $f(x_1,...,x_n)$ transforms into 
\begin{equation*}
f({u_1 \over u_i} x_i, ... ,{u_n \over u_i} x_i) = x_i^d \cdot g({u_1 \over u_i} x_i, ... ,{u_n \over u_i} x_i),
\end{equation*}
where $g$ is not divisible by $x_i$. As the variety in question is isomorphic to $\Aff^{n}$ with coordinates $x_i, u_1/u_i,...,u_n/u_i$, we can again conclude that the Zariski closure in the $i^{th}$ chart is cut out by the polynomial $g$. Therefore one can compute the strict transform $X^{st}$ of the hypersurface $X$ simply by finding the local equations for all the $n$ charts. Finding the strict transform of non-hypersurfaces is harder, as finding the Zariski closure is not as simple, not even locally.
\end{ex}

\begin{ex}
Assume that we have a plane curve $C$, cut out by polynomial $f$, and that the origin is a simple singularity of degree $d$. This means that at the origin we have $d$ branches of the curve meeting transversally, i.e., their tangent lines are distinct. The lowest degree form $F_{min}$ of $f$ factors into a linear product $\prod_i (a_i x_i - b_i y_i)$ and the lines defined by these linear factors are known to be exactly the tangent lines of the branches at the origin. Hence, the degree of the form is $d$, and the linear factors are pairwisely different.

Blowing up this kind of singularity resolves it immediately. We break the polynomial $f$ into a sum $F_d + F_{d+1} + ... + F_n$ of its homogeneous parts, the reader should note that $F_d = F_{min}$. We can see that the polynomial $x^{-d} f(x,xz)$ defining the strict transformation is now 
\begin{equation*}
F_d(1,z) + x F_{d+1} (1,z) + ... + x^{n-d} F_n (1,z),
\end{equation*}
from which we see that the points of the strict transform $C^{st}$ lying over the origin in the first chart are exactly the ones having $x=0$ and $z$ satisfying $F_d(1,z)$, which by assumption splits into \emph{distinct} linear factors. As the intersection multiplicity with $x=0$ is 1 at each of these points, we see that the points must be simple. In the other chart one does exactly the same things, and by symmetry we can conclude that all points lying over the origin in the strict transformation $X^{st}$ are simple.
\end{ex}

\begin{ex}
A single blow-up may not be able to resolve the singularity even in the case of plane curves. Take the curve $y^2 - x^5 = 0$. The strict transform of in the first chart is cut out by $z^2 - x^3$, which has cusp at the origin. This could be resolved by blowing up the origin again (which we strictly speaking don't know how to do yet, because the strict transform is not necessarily contained in a affine space). 
\end{ex}

\begin{ex}
Sometimes we cannot resolve a singular point no matter how many times we blow it up. The famous example of this is the \emph{Whitney umbrella} which is an algebraic surface in $\Aff^3$ cut out by $x^2 - y^2 z$. We compute the strict transform on the $z$-chart with coordinates $x',y',z$ where $x=x'z$ and $x=x'z$, and see that it is cut out by the equation $z^{-2} (x'^2 z^2 - y'^2z^3) = x'^2 - y'^2z$, which is essentially the same equation! Therefore blowing up the origin can never resolve the singularity. The singularity of the Whitney umbrella can however be resolved by blowing up, but instead of blowing up a point, we need to blow up the whole $z$-axis.
\end{ex}

\subsection{Blowing up simple subvarities}

In some cases it is very easy to see what \emph{blowing up a subvariety} $Y$ of $\Aff^n$ should mean. Take for example the $z$-axis $Z$ in $\Aff^3$. We can think about $\Aff^3$ as an infinite family of affine $2$-spaces parametrized by the $z$-coordinate, and we interpret blowing up the $z$-axis (almost) just as blowing up the origin at each of these planes simultaneously. We end up with the blow-up map $\bl (\Aff^n) \times \Aff^1 \cong \bl_Z(\Aff^3) \to \Aff^3$. The strict and total transforms of subvarities are defined similarly as before.

\begin{ex}
We can now resolve the Whitney umbrella, which is cut out by $x^2 - y^2 z$. The coordinates for $\bl_Z (A^3)$ are $(x,y,z,[u:v])$, and the relation $xv-yu = 0$ is satisfied. On the first chart, we have 
\begin{equation*}
y = x {v \over u},
\end{equation*} 
and hence the defining equation can be transformed into
\begin{equation*}
x^2 -  x^2\left( {v \over u} \right)^2 z = x^2 ( 1 - \left( {v \over u} \right)^2 z ).
\end{equation*}
Therefore the first chart gives us the equation $u^2 - v^2 z = 0$. Similarly we can calculate the strict transform in the second chart, but we get exactly the same equation from there. Hence the strict transform of the Whitney umbrella is the closed subvariety of $\Aff^3 \times \Proj^1$ cut out by the equations
\begin{align*}
xv-yu &= 0\\
u^2 - v^2 z &= 0.
\end{align*}
Moreover, we can see from the local equations that the strict transform is nonsingular, so we have managed to resolve the Whitney umbrella.
\end{ex}

\subsection{Blowing up, the general construction}

\section{Hironaka's Proof}\label{HirRes}

\section{de Jong's Proof}\label{deJongRes}

\begin{thebibliography}{5, style=alpha}

\bibitem[AtM]{AtM} 
Michael Atiyah and Ian Macdonald: Introduction to Commutative Algebra, Addison-Wesley.

\bibitem[Cut]{Cut} 
Steven Dale Cutkosky: Resolution of Singularities, Graduate Studies in Mathematics 63, 2004.

\bibitem[Eis]{Eis}
David Eisenbud: Commutative Algebra with a View Toward Algebraic Geometry, Springer-Verlag.

\bibitem[Har]{Har}
Robin Hartshorne: Algebraic Geometry, 1st edition, Springer, 1977.

\bibitem[Hau]{Hau}
Herwig Hauser: The Hironaka Theorem on Resolution of Singularities (Or: A proof we always wanted to understand) Bull. Amer. Math. Soc. 40 (2003).

\bibitem[HLOQ]{HLOQ} 
Hauser, Lipman, Oort, Quirós: Resolution of Singularities - A research textbook in tribute to Oscar Zariski, Progress in Mathematics 181, Birkhäuser Verlag, 2000.

\bibitem[Jong]{Jong}
Aise Johan de Jong: Smoothness, semi-stability and alterations, Publications Mathématiques de l’Institut des Hautes Études Scientifiques (1996).

\bibitem[Kol]{Kol}
János Kollár: Lectures on Resolution of Singularities, Annals of Mathematics Studies 166, 2007.

\bibitem[Mum]{Mum}
David Mumford, Selected papers. Volume II. On algebraic geometry, including correspondence with Grothendieck. Springer-Verlag, 2010.

\bibitem[Par]{Par} Carol Parkih: The Unreal Life of Oscar Zariski, Academic Press Inc, 1991.

\bibitem[Ser]{Ser}
Jean-Pierre Serre: Local Algebra, Springer, 2000.

\bibitem[Vak]{Vak}
Ravi Vakil: The Rising Sea - Foundations of Algebraic Geometry.

\bibitem[W\l{}o]{Wlo}
Jaros\l{}aw W\l{}odarczyk: Simple Hironaka Resolution in Characteristic Zero, J. Amer. Math. Soc. 18 (2005)
\end{thebibliography}


\end{document}
